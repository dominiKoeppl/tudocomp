%% IMPORT-DATA tudocomp_memory log/tudocomp_memory.log
%% IMPORT-DATA tudocomp_time log/tudocomp_time.log
%% IMPORT-DATA low log/low.log
%% IMPORT-DATA randomtime log/randomtime.log
%% IMPORT-DATA randomspace log/randomspace.log
%% IMPORT-DATA unixcompress log/unixcompress.log

\documentclass[a4paper]{article}
\usepackage{fullpage}

\usepackage[group-separator={,}]{siunitx}
\usepackage{booktabs}
\usepackage{adjustbox}


\newcommand*{\namelabel}[1]{{\textsc{#1}}}
\newcommand*{\LZSeven}{{\namelabel{lz77}}}
\newcommand*{\LZW}{{\namelabel{lzw}}}
\newcommand*{\LZEight}{{\namelabel{lz78}}}
\newcommand*{\LZTrie}{{\textsc{lz} trie}}

\newcommand*{\instancename}[1]{\ensuremath{\mathsf{#1}}} %for instance names
\DeclareMathAlphabet{\mathup}{OT1}{\familydefault}{m}{n}
\RequirePackage{mathtools}
\DeclarePairedDelimiter\bsq{\text{\lq}}{\text{\rq}}     % '.', ticks for '1'/'0'

\newcommand*\xor{\ensuremath{\mathbin{\oplus}}}
\newcommand{\lcg}[1]{\ensuremath{\instancename{lcg}_{#1}}}
\newcommand{\sxor}[1]{\instancename{sxor}_{#1}}
\newcommand{\id}{\instancename{id}}

\newcommand{\select}[1][]{\operatorname{select}_{#1}}
\newcommand{\rank}[1][]{\operatorname{rank}_{#1}}

\newcommand*{\iBucket}{\textsf{bucket}}
\newcommand*{\iUnordered}[1]{\ensuremath{\textsf{unordered}_{\textup{#1}}}}
\newcommand*{\iGrp}[1]{\ensuremath{\textsf{grp}_{#1}}}
\newcommand*{\iCleary}[1]{\ensuremath{\textsf{cleary}_{\textup{#1}}}}
\newcommand*{\iElias}[1]{\ensuremath{\textsf{elias}_{\textup{#1}}}}
\newcommand*{\iLayered}[1]{\ensuremath{\textsf{layered}_{\textup{#1}}}}


\newcommand*{\JumpPointer}[1]{\ensuremath{#1^{\mathup{J}}}}

\newcommand*{\iFix}[1]{\ensuremath{\textsf{fix}_{\textup{#1}}}}
\newcommand*{\iMulti}[1]{\ensuremath{\textsf{multi}_{\textup{#1}}}}
\newcommand*{\iGrow}[1]{\ensuremath{\textsf{grow}_{\textup{#1}}}}
\newcommand*{\iClassic}[1]{\ensuremath{\textsf{classic}_{\textup{#1}}}}
\newcommand*{\iBrute}[1]{\ensuremath{\textsf{brute}_{\textup{#1}}}}


\newcommand{\iRolling}[1]{\ensuremath{\instancename{rolling_{#1}}}}
\newcommand{\iCompact}[1]{\ensuremath{\instancename{cht}_{#1}}}
\newcommand{\iHash}[1]{\ensuremath{\instancename{hash_{#1}}}}
\newcommand{\iJudy}   {\instancename{judy}}
\newcommand{\iTernary}{\instancename{ternary}}
\newcommand{\iBinary}[1] {\ensuremath{\instancename{binary}_{\textup{#1}}}}

\newcommand*{\hashfunction}[1]{{\usefont{T1}{ppl}{m}{sl}#1}}
\newcommand*{\vFermat} {\hashfunction{fermat}}
\newcommand*{\vIDThree}{\hashfunction{ID37}}

\newcommand*{\bv}[1]{\ensuremath{B_{\mathup{#1}}}}
\newcommand*{\bvJ}[1]{\ensuremath{B_{#1}}}

\newcommand*{\setKeys}{\ensuremath{\mathcal{K}}}
\newcommand*{\fnLookup}{\functionname{lookup}}
\newcommand*{\fnInsert}{\functionname{insert}}
\newcommand*{\fnDepth}{\functionname{depth}}
\newcommand*{\fnRoot}{\functionname{root}}

\newcommand*{\fnKey}{\functionname{key}}
\newcommand*{\timeBonsai}{\ensuremath{t_{\mathup{Bonsai}}}}

\newcommand*{\runningExample}{aaababaaaba}

\newcommand*{\Dataset}[1]{\textsc{#1}}

\usepackage{tikz}
\usetikzlibrary{matrix,positioning,fit}
\usetikzlibrary{arrows}


%%%% BEGIN TIKZ PGFPLOTS
\usepackage{tikz}
% \usetikzlibrary{external}
% \tikzexternalize[prefix=plot/]
\usepackage{pgfplots}
\pgfplotsset{compat=1.16}
\usepgfplotslibrary{colorbrewer}

\pgfplotsset{%
% initialize Dark2
cycle list/Dark2,
cycle multiindex* list={%
mark list*\nextlist
Dark2\nextlist
},
}
%Dark2 has size 8
%Mark List has size 11


\pgfplotsset{%
squeezedPlot/.style={%
width=0.5\linewidth,
height=0.5\linewidth
},
appendLegend/.style={%
legend style={font=\small}},
bitsXaxis/.style={%
    extra x ticks={8},
grid=minor,
    extra tick style={tick style={draw=none},ticklabel pos=top, tick label style={color=gray,font=\tiny},grid=major, grid style={dotted, gray}},
	%extra x tick labels={},
	},
	tinyX/.style={%
		scaled x ticks=false,
		x tick label style={%
			/pgf/number format/.cd,
			fixed,
			fixed zerofill,
			precision=3,
			/tikz/.cd
		},
	},
	tinyY/.style={%
		scaled y ticks=false,
		y tick label style={%
			/pgf/number format/.cd,
			fixed,
			fixed zerofill,
			precision=3,
			/tikz/.cd
		},
	},
}

\newcommand{\PlotLarge}[3][]{%
	\begin{minipage}{0.49\linewidth}
\begin{tikzpicture}
	\begin{axis}[title={\textsc{#2}},#1]
		#3
\end{axis}
\end{tikzpicture}
	\end{minipage}
}
\newcommand{\PlotSmall}[3][]{%
	\begin{minipage}{0.24\linewidth}
\begin{tikzpicture}
	\begin{axis}[title={\textsc{#2}},#1]
		#3
\end{axis}
\end{tikzpicture}
	\end{minipage}
}

\begin{document}

\begin{table}
\centerline{%
\begin{tabular}{l*{6}{r}}
\toprule
text & $n$ [M] & $\sigma$ & $z_{\LZEight}$ [M] & $c_{\LZEight}$ [MB] & $z_{\LZW}$ [M] & $c_{\LZW}$ [MB]
\\\midrule
%% TABULAR 
%% SELECT '\Dataset{' || `78`.file || '}', 
%% printf("%.2f", `78`.n/1e6), max(low.sigma), 
%% printf("%.2f", `78`.`z`/1e6), 
%% printf("%.2f", `78`.`compressedsize`/1e6), 
%% printf("%.2f", w.z/1e6), 
%% printf("%.2f", w.compressedsize/1e6)
%% FROM tudocomp_memory AS `78` 
%% JOIN low ON low.file = `78`.file 
%% JOIN tudocomp_memory AS w ON w.file = `78`.file AND w.action = `78`.action and w.algo = `78`.algo 
%% WHERE `78`.action='compression' and `78`.type = '78' and w.type ='w' and `78`.algo = 'binary' 
%% AND `78`.`file` NOT IN ('english.1')
%% GROUP BY `78`.file
%% CONFIG file=plot/tabDatasets.tex
\csname @@input\endcsname {plot/tabDatasets.tex}
%
			\bottomrule
		\end{tabular}
	}%centerline
\caption{Text files used in the experiments.
	Columns marked with $\bsq{\textup{[M]}}$ are divided by $10^{-6}$;
	$z_{\LZEight}$ and $z_{\LZW}$ are the number of factors of the \LZEight{} and \LZW{} factorization, respectively;
	$c_{\LZEight}$ and $c_{\LZW}$ are the size of the encoding of the \LZEight{} and the \LZW{} factors.
}
\label{tabDatasets}
\end{table}

\begin{figure}[t]
\pgfplotsset{squeezedPlot}

\begin{tikzpicture}
\begin{axis}[
title={Insertion Time},
xlabel={number of elements [$\lg$]},
ylabel={avg.\ time per element [$\mu$s/\#]},
appendLegend,
legend to name={legRandomCompactEight},
legend columns=2
]


%% MULTIPLOT(experiment) SELECT log(2,size) AS x, time*1000.0/size as y, MULTIPLOT FROM randomtime
%% WHERE `group` = "insert"
%% AND `value_width`='8'
%% AND size between 1048576 AND 134217728
%% AND experiment IN ('clearyS', 'clearyP', 'layeredS', 'layeredP', 'chtD', 'grp', 'eliasS', 'eliasP')
%% GROUP BY MULTIPLOT,x ORDER BY MULTIPLOT,x
%% CONFIG file=plot/cht_construction.tex type=tex
\input{plot/cht_construction.tex}
\end{axis}
\end{tikzpicture}
\begin{tikzpicture}
\begin{axis}[
title={Query Time},
xlabel={number of elements [$\lg$]},
ylabel={avg.\ time per element [$\mu$s/\#]}
]

%% MULTIPLOT(experiment) SELECT LOG(2,size) AS x, time*1000.0/size as y, MULTIPLOT FROM randomtime
%% WHERE `group` = "query"
%% AND `value_width`='8'
%% AND size between 1048576 AND 134217728
%% AND experiment IN ('clearyS', 'clearyP', 'layeredS', 'layeredP', 'chtD', 'grp', 'eliasS', 'eliasP')
%% GROUP BY MULTIPLOT,x ORDER BY MULTIPLOT,x
%% CONFIG file=plot/cht_query.tex type=tex
\input{plot/cht_query.tex}
\legend{}
\end{axis}
\end{tikzpicture}

\begin{adjustbox}{valign=t}
\begin{tikzpicture}
\begin{axis}[
title={Space},
xlabel={number of elements [$\lg$]},
ylabel={avg.\ memory per element [bytes/\#]}
]

%% MULTIPLOT(experiment)
%% SELECT LOG(2,size) AS x, mem*1.0/size as y, MULTIPLOT FROM randomspace
%% WHERE `group` = 'insert' 
%% AND `value_width`='8'
%% AND size between 1048576 AND 134217728
%% AND experiment IN ('clearyS', 'clearyP', 'layeredS', 'layeredP', 'chtD', 'grp', 'eliasS', 'eliasP')
%% GROUP BY MULTIPLOT,x ORDER BY MULTIPLOT,x
%% CONFIG file=plot/cht_memory.tex type=tex
\input{plot/cht_memory.tex}
\legend{}
\end{axis}
\end{tikzpicture}
\end{adjustbox}
\begin{adjustbox}{valign=t}
\begin{minipage}{0.5\linewidth}
\begin{center}
\ref{legRandomCompactEight}
\end{center}

\emph{Top Left}: Time for inserting $2^{10} \cdot (3/2)^n$ elements for $n \ge 0$.
\emph{Top Right:} Time for querying all inserted elements.
\emph{Bottom Left}: Maximum space needed during the insertions.
\end{minipage}
\end{adjustbox}
\caption{%
	Managing randomly generated 32-bit keys and 8-bit values with the compact hash tables.
}
\label{figRandomCompactEight}
\end{figure}

\begin{figure}[t!]

%omit: 'compact_kv', 'compact_kvP',
\pgfplotsset{squeezedPlot,
	xlabel style={align=center,yshift=-1em},
	xlabel={memory\\ $[$bits/$n]$},
	xticklabel style={rotate=90, anchor=east},
	ticklabel style={font=\small},
	width=\linewidth,
	height=\linewidth,
	y tick label style={%
		/pgf/number format/.cd,
		fixed relative,
		precision=2,
		zerofill,
		/tikz/.cd,
	},
}

%% DEFINE 78plot(file)
%% SELECT m.mem*8.0/m.n as x, MIN(t.time*1e6/t.n) as y, MULTIPLOT from tudocomp_memory AS m 
%% JOIN tudocomp_time AS t ON m.algo = t.algo AND t.type = m.type AND t.file = m.file AND t.action = m.action
%% WHERE m.type='78' AND m.file=$file AND m.action='compression' 
%% AND m.algo in ('binary', 'binaryMTF', 'binarysorted', 'binaryP', 'binaryk', 'binaryMTF', 'binarykP')
%% GROUP BY MULTIPLOT,x ORDER BY MULTIPLOT,x

\PlotSmall[ylabel={time [$\mu\text{s}/n$]},
xlabel=,
appendLegend,
legend to name={legPlotSevenEightBinaryOuttakes},
legend columns=5]{dna}{%

%% MULTIPLOT(m.algo) $78plot('dna')
%% CONFIG file=plot/78_binaryOuttake_dna.tex type=tex
\input{plot/78_binaryOuttake_dna.tex}
}%Plot
%
\PlotSmall[xlabel=]{english}{%
%% MULTIPLOT(m.algo) $78plot('english')
%% CONFIG file=plot/78_binaryOuttake_english.tex type=tex
\input{plot/78_binaryOuttake_english.tex}
\legend{}
}%Plot
%
\PlotSmall[xlabel=]{proteins}{%
%% MULTIPLOT(m.algo) $78plot('proteins')
%% CONFIG file=plot/78_binaryOuttake_proteins.tex type=tex
\input{plot/78_binaryOuttake_proteins.tex}
\legend{}
}%Plot
%
\PlotSmall[xlabel=]{xml}{%
%% MULTIPLOT(m.algo) $78plot('xml')
%% CONFIG file=plot/78_binaryOuttake_xml.tex type=tex
\input{plot/78_binaryOuttake_xml.tex}
\legend{}
}%Plot

\PlotSmall[bitsXaxis, ylabel={time [$\mu\text{s}/n$]}]{commoncrawl}{%
%% MULTIPLOT(m.algo) $78plot('commoncrawl')
%% CONFIG file=plot/78_binaryOuttake_commoncrawl.tex type=tex
\input{plot/78_binaryOuttake_commoncrawl.tex}
\legend{}
}%Plot
%
\hspace{1em}
\PlotSmall[tinyX,tinyY]{fibonacci}{%
%% MULTIPLOT(m.algo) $78plot('fibonacci')
%% CONFIG file=plot/78_binaryOuttake_fibonacci.tex type=tex
\input{plot/78_binaryOuttake_fibonacci.tex}
\legend{}
}%Plot
\hspace{-1em}
%
\PlotSmall{gutenberg}{%
%% MULTIPLOT(m.algo) $78plot('gutenberg')
%% CONFIG file=plot/78_binaryOuttake_gutenberg.tex type=tex
\input{plot/78_binaryOuttake_gutenberg.tex}
\legend{}
}%Plot
%
\PlotSmall{wikipedia}{%
%% MULTIPLOT(m.algo) $78plot('wikipedia')
%% CONFIG file=plot/78_binaryOuttake_wikipedia.tex type=tex
\input{plot/78_binaryOuttake_wikipedia.tex}
\legend{}
}%Plot

%% UNDEF 78plot(file)
\ref{legPlotSevenEightBinaryOuttakes}

%% DEFINE 78plot(file)
%% SELECT m.mem*8.0/m.n as x, MIN(t.time*1e6/t.n) as y, MULTIPLOT from tudocomp_memory AS m 
%% JOIN tudocomp_time AS t ON m.algo = t.algo AND t.type = m.type AND t.file = m.file AND t.action = m.action
%% WHERE m.type='78' AND m.file=$file AND m.action='compression' 
%% AND m.algo in ('hash', 'exthash', 'chain0', 'chain0P')
%% GROUP BY MULTIPLOT,x ORDER BY MULTIPLOT,x

\PlotSmall[bitsXaxis,ylabel={time [$\mu\text{s}/n$]},
xlabel=,
appendLegend,
legend to name={legPlotSevenEightHashOuttakes},
legend columns=5]{dna}{%

%% MULTIPLOT(m.algo) $78plot('dna')
%% CONFIG file=plot/78_hashOuttake_dna.tex type=tex
\input{plot/78_hashOuttake_dna.tex}
}%Plot
%
\PlotSmall[bitsXaxis,xlabel=]{english}{%
%% MULTIPLOT(m.algo) $78plot('english')
%% CONFIG file=plot/78_hashOuttake_english.tex type=tex
\input{plot/78_hashOuttake_english.tex}
\legend{}
}%Plot
%
\PlotSmall[bitsXaxis,xlabel=]{proteins}{%
%% MULTIPLOT(m.algo) $78plot('proteins')
%% CONFIG file=plot/78_hashOuttake_proteins.tex type=tex
\input{plot/78_hashOuttake_proteins.tex}
\legend{}
}%Plot
%
\PlotSmall[bitsXaxis,xlabel=]{xml}{%
%% MULTIPLOT(m.algo) $78plot('xml')
%% CONFIG file=plot/78_hashOuttake_xml.tex type=tex
\input{plot/78_hashOuttake_xml.tex}
\legend{}
}%Plot

\PlotSmall[bitsXaxis,ylabel={time [$\mu\text{s}/n$]}]{commoncrawl}{%
%% MULTIPLOT(m.algo) $78plot('commoncrawl')
%% CONFIG file=plot/78_hashOuttake_commoncrawl.tex type=tex
\input{plot/78_hashOuttake_commoncrawl.tex}
\legend{}
}%Plot
%
\PlotSmall{fibonacci}{%
%% MULTIPLOT(m.algo) $78plot('fibonacci')
%% CONFIG file=plot/78_hashOuttake_fibonacci.tex type=tex
\input{plot/78_hashOuttake_fibonacci.tex}
\legend{}
}%Plot
%
\PlotSmall[bitsXaxis]{gutenberg}{%
%% MULTIPLOT(m.algo) $78plot('gutenberg')
%% CONFIG file=plot/78_hashOuttake_gutenberg.tex type=tex
\input{plot/78_hashOuttake_gutenberg.tex}
\legend{}
}%Plot
%
\PlotSmall[bitsXaxis]{wikipedia}{%
%% MULTIPLOT(m.algo) $78plot('wikipedia')
%% CONFIG file=plot/78_hashOuttake_wikipedia.tex type=tex
\input{plot/78_hashOuttake_wikipedia.tex}
\legend{}
}%Plot

%% UNDEF 78plot(file)
\ref{legPlotSevenEightHashOuttakes}

\caption{Evaluation of the \LZEight{} factorization with the trie implementations,
	namely
	\iBinary{mtf} (\iBinary{} using MTF-encoding),
	\iBinary{k} (\iBinary{},
	\JumpPointer{\iBinary{k}} (\iBinary{k} combined with the jump pointer technique,
	\iBinary{s} (\iBinary{} with sorting), 
	\JumpPointer{\iGrp{0}} (jump pointers applied to \iGrp{0} for $\beta = 0$),
	and
	\iUnordered{} (\iHash{} with the C++ STL hash table), 
	which we omit in the following evaluations.
}
\label{figPlotSevenEightOuttakes}
\end{figure}

\begin{figure}[t!]
%% DEFINE 78plot(file)
%% SELECT m.mem*8.0/m.n as x, MIN(t.time*1e6/t.n) as y, MULTIPLOT from tudocomp_memory AS m 
%% JOIN tudocomp_time AS t ON m.algo = t.algo AND t.type = m.type AND t.file = m.file AND t.action = m.action
%% WHERE m.type='78' AND m.file=$file AND m.action='compression' 
%% AND m.algo not in ('rolling128', 'rolling128plus', 'binaryk', 'binaryMTF', 'binarykP', 'binaryMTF', 'chain0P', 'compact_kvP', 'exthash', 'binarysorted')
%% GROUP BY MULTIPLOT,x ORDER BY MULTIPLOT,x

\pgfplotsset{bitsXaxis,
	width=\linewidth,
	height=0.6\linewidth,
	ylabel={time [$\mu\text{s}/n$]},
	xlabel={memory $[$bits/$n]$},
	y tick label style={%
		/pgf/number format/.cd,
		fixed relative,
		precision=2,
		zerofill,
		/tikz/.cd,
	},
	ymin=0,
}



\PlotLarge[xlabel=,appendLegend,
legend to name={legPlotSevenEightAll},
legend columns=5]{dna}{%

%% MULTIPLOT(m.algo) $78plot('dna')
%% CONFIG file=plot/78_all_dna.tex type=tex
\input{plot/78_all_dna.tex}
}%Plot
\hfill
%
\PlotLarge[xlabel=,ylabel=]{english}{%
%% MULTIPLOT(m.algo) $78plot('english')
%% CONFIG file=plot/78_all_english.tex type=tex
\input{plot/78_all_english.tex}
\legend{}
}%Plot

\PlotLarge[xlabel=]{proteins}{%
%% MULTIPLOT(m.algo) $78plot('proteins')
%% CONFIG file=plot/78_all_proteins.tex type=tex
\input{plot/78_all_proteins.tex}
\legend{}
}%Plot
\hfill
\PlotLarge[xlabel=,ylabel=]{xml}{%
%% MULTIPLOT(m.algo) $78plot('xml')
%% CONFIG file=plot/78_all_xml.tex type=tex
\input{plot/78_all_xml.tex}
\legend{}
}%Plot

\PlotLarge[xlabel=]{commoncrawl}{%
%% MULTIPLOT(m.algo) $78plot('commoncrawl')
%% CONFIG file=plot/78_all_commoncrawl.tex type=tex
\input{plot/78_all_commoncrawl.tex}
\legend{}
}%Plot
\hfill
\PlotLarge[xlabel=,ylabel=]{fibonacci}{%
%% MULTIPLOT(m.algo) $78plot('fibonacci')
%% CONFIG file=plot/78_all_fibonacci.tex type=tex
\input{plot/78_all_fibonacci.tex}
\legend{}
}%Plot

\PlotLarge{gutenberg}{%
%% MULTIPLOT(m.algo) $78plot('gutenberg')
%% CONFIG file=plot/78_all_gutenberg.tex type=tex
\input{plot/78_all_gutenberg.tex}
\legend{}
}%Plot
\hfill
\PlotLarge[ylabel=]{wikipedia}{%
%% MULTIPLOT(m.algo) $78plot('wikipedia')
%% CONFIG file=plot/78_all_wikipedia.tex type=tex
\input{plot/78_all_wikipedia.tex}
\legend{}
}%Plot


%% UNDEF 78plot(file)
\ref{legPlotSevenEightAll}

\caption{Juxtaposition of all \LZTrie{} implementations for the \LZEight{} factorization.
}
\label{figPlotSevenEightAll}
\end{figure}




\begin{figure}[t!]
%% DEFINE 78plot(file)
%% SELECT m.mem*8.0/m.n as x, MIN(t.time*1e6/t.n) as y, MULTIPLOT from tudocomp_memory AS m 
%% JOIN tudocomp_time AS t ON m.algo = t.algo AND t.type = m.type AND t.file = m.file AND t.action = m.action
%% WHERE m.type='78' AND m.file=$file AND m.action='compression' 
%% AND m.algo in ('hash', 'hashplus', 'rolling', 'rollingplus', 'binary', 'ternary', 'judy')
%% GROUP BY MULTIPLOT,x ORDER BY MULTIPLOT,x

\pgfplotsset{squeezedPlot,
	xlabel style={align=center,yshift=-1em},
	xlabel={memory\\ $[$bits/$n]$},
	xticklabel style={rotate=90, anchor=east},
	ticklabel style={font=\small},
	width=\linewidth,
	height=\linewidth,
	y tick label style={%
		/pgf/number format/.cd,
		fixed relative,
		precision=2,
		zerofill,
		/tikz/.cd,
	},
}



\PlotSmall[ylabel={time [$\mu\text{s}/n$]},
bitsXaxis,
xlabel=,
appendLegend,
legend to name={legPlotSevenEightHighMem},
legend columns=5]{dna}{%

%% MULTIPLOT(m.algo) $78plot('dna')
%% CONFIG file=plot/78_highmem_dna.tex type=tex
\input{plot/78_highmem_dna.tex}
}%Plot
%
\PlotSmall[bitsXaxis,xlabel=]{english}{%
%% MULTIPLOT(m.algo) $78plot('english')
%% CONFIG file=plot/78_highmem_english.tex type=tex
\input{plot/78_highmem_english.tex}
\legend{}
}%Plot
%
\PlotSmall[bitsXaxis,xlabel=]{proteins}{%
%% MULTIPLOT(m.algo) $78plot('proteins')
%% CONFIG file=plot/78_highmem_proteins.tex type=tex
\input{plot/78_highmem_proteins.tex}
\legend{}
}%Plot
%
\PlotSmall[bitsXaxis,xlabel=]{xml}{%
%% MULTIPLOT(m.algo) $78plot('xml')
%% CONFIG file=plot/78_highmem_xml.tex type=tex
\input{plot/78_highmem_xml.tex}
\legend{}
}%Plot

\PlotSmall[bitsXaxis,ylabel={time [$\mu\text{s}/n$]}]{commoncrawl}{%
%% MULTIPLOT(m.algo) $78plot('commoncrawl')
%% CONFIG file=plot/78_highmem_commoncrawl.tex type=tex
\input{plot/78_highmem_commoncrawl.tex}
\legend{}
}%Plot
%
\PlotSmall{fibonacci}{%
%% MULTIPLOT(m.algo) $78plot('fibonacci')
%% CONFIG file=plot/78_highmem_fibonacci.tex type=tex
\input{plot/78_highmem_fibonacci.tex}
\legend{}
}%Plot
%
\PlotSmall[bitsXaxis]{gutenberg}{%
%% MULTIPLOT(m.algo) $78plot('gutenberg')
%% CONFIG file=plot/78_highmem_gutenberg.tex type=tex
\input{plot/78_highmem_gutenberg.tex}
\legend{}
}%Plot
%
\PlotSmall[bitsXaxis]{wikipedia}{%
%% MULTIPLOT(m.algo) $78plot('wikipedia')
%% CONFIG file=plot/78_highmem_wikipedia.tex type=tex
\input{plot/78_highmem_wikipedia.tex}
\legend{}
}%Plot

%% UNDEF 78plot(file)
\ref{legPlotSevenEightHighMem}

%% DEFINE 78plot(file)
%% SELECT m.mem*8.0/m.n as x, MIN(t.time*1e6/t.n) as y, MULTIPLOT from tudocomp_memory AS m 
%% JOIN tudocomp_time AS t ON m.algo = t.algo AND t.type = m.type AND t.file = m.file AND t.action = m.action
%% WHERE m.type='78' AND m.file=$file AND m.action='compression' 
%% AND m.algo not in ('hash', 'hashplus', 'rolling', 'rollingplus', 'rolling128', 'rolling128plus', 'binary', 'ternary', 'judy', 'binaryk', 'binaryMTF', 'binarykP', 'chain0P', 'compact_kvP', 'binarysorted', 'exthash')
%% GROUP BY MULTIPLOT,x ORDER BY MULTIPLOT,x

\PlotSmall[bitsXaxis,ylabel={time [$\mu\text{s}/n$]},
appendLegend,
xlabel=,
legend to name={legPlotSevenEight},
legend columns=5]{dna}{%
%% MULTIPLOT(m.algo) $78plot('dna')
%% CONFIG file=plot/78_dna.tex type=tex
\input{plot/78_dna.tex}
}%Plot
%
\PlotSmall[bitsXaxis,xlabel=]{english}{%
%% MULTIPLOT(m.algo) $78plot('english')
%% CONFIG file=plot/78_english.tex type=tex
\input{plot/78_english.tex}
\legend{}
}%Plot
%
\PlotSmall[bitsXaxis,xlabel=]{proteins}{%
%% MULTIPLOT(m.algo) $78plot('proteins')
%% CONFIG file=plot/78_proteins.tex type=tex
\input{plot/78_proteins.tex}
\legend{}
}%Plot
%
\PlotSmall[bitsXaxis,xlabel=]{xml}{%
%% MULTIPLOT(m.algo) $78plot('xml')
%% CONFIG file=plot/78_xml.tex type=tex
\input{plot/78_xml.tex}
\legend{}
}%Plot

\PlotSmall[bitsXaxis,ylabel={time [$\mu\text{s}/n$]}]{commoncrawl}{%
%% MULTIPLOT(m.algo) $78plot('commoncrawl')
%% CONFIG file=plot/78_commoncrawl.tex type=tex
\input{plot/78_commoncrawl.tex}
\legend{}
}%Plot
%
\PlotSmall[tinyX]{fibonacci}{%
%% MULTIPLOT(m.algo) $78plot('fibonacci')
%% CONFIG file=plot/78_fibonacci.tex type=tex
\input{plot/78_fibonacci.tex}
\legend{}
}%Plot
%
\PlotSmall[bitsXaxis]{gutenberg}{%
%% MULTIPLOT(m.algo) $78plot('gutenberg')
%% CONFIG file=plot/78_gutenberg.tex type=tex
\input{plot/78_gutenberg.tex}
\legend{}
}%Plot
%
\PlotSmall[bitsXaxis]{wikipedia}{%
%% MULTIPLOT(m.algo) $78plot('wikipedia')
%% CONFIG file=plot/78_wikipedia.tex type=tex
\input{plot/78_wikipedia.tex}
\legend{}
}%Plot

\ref{legPlotSevenEight}

\caption{Evaluation of our \LZTrie{} implementations for the \LZEight{} factorization.
}
\label{figPlotSevenEightDivided}
\end{figure}

\begin{figure}[t!]
%% DEFINE wplot(file)
%% SELECT m.mem*8.0/m.n as x, MIN(t.time*1e6/t.n) as y, MULTIPLOT from tudocomp_memory AS m 
%% JOIN tudocomp_time AS t ON m.algo = t.algo AND t.type = m.type AND t.file = m.file AND t.action = m.action
%% WHERE m.type='w' AND m.file=$file AND m.action='compression' 
%% AND m.algo not in ('rolling128', 'rolling128plus', 'binaryk', 'binaryMTF', 'binarykP', 'chain0P', 'compact_kvP', 'binarysorted', 'exthash')
%% GROUP BY MULTIPLOT,x ORDER BY MULTIPLOT,x

\pgfplotsset{bitsXaxis,
	width=\linewidth,
	height=0.6\linewidth,
	ylabel={time [$\mu\text{s}/n$]},
	xlabel={memory $[$bits/$n]$},
	y tick label style={%
		/pgf/number format/.cd,
		fixed relative,
		precision=2,
		zerofill,
		/tikz/.cd,
	},
	ymin=0,
}


\PlotLarge[xlabel=,appendLegend,
legend to name={legPlotWelchAll},
legend columns=5]{dna}{%

%% MULTIPLOT(m.algo) $wplot('dna')
%% CONFIG file=plot/w_all_dna.tex type=tex
\input{plot/w_all_dna.tex}
}%Plot
\hfill
%
\PlotLarge[xlabel=,ylabel=]{english}{%
%% MULTIPLOT(m.algo) $wplot('english')
%% CONFIG file=plot/w_all_english.tex type=tex
\input{plot/w_all_english.tex}
\legend{}
}%Plot

\PlotLarge[xlabel=]{proteins}{%
%% MULTIPLOT(m.algo) $wplot('proteins')
%% CONFIG file=plot/w_all_proteins.tex type=tex
\input{plot/w_all_proteins.tex}
\legend{}
}%Plot
\hfill
\PlotLarge[xlabel=,ylabel=]{xml}{%
%% MULTIPLOT(m.algo) $wplot('xml')
%% CONFIG file=plot/w_all_xml.tex type=tex
\input{plot/w_all_xml.tex}
\legend{}
}%Plot

\PlotLarge[xlabel=]{commoncrawl}{%
%% MULTIPLOT(m.algo) $wplot('commoncrawl')
%% CONFIG file=plot/w_all_commoncrawl.tex type=tex
\input{plot/w_all_commoncrawl.tex}
\legend{}
}%Plot
\hfill
\PlotLarge[xlabel=,ylabel=]{fibonacci}{%
%% MULTIPLOT(m.algo) $wplot('fibonacci')
%% CONFIG file=plot/w_all_fibonacci.tex type=tex
\input{plot/w_all_fibonacci.tex}
\legend{}
}%Plot

\PlotLarge{gutenberg}{%
%% MULTIPLOT(m.algo) $wplot('gutenberg')
%% CONFIG file=plot/w_all_gutenberg.tex type=tex
\input{plot/w_all_gutenberg.tex}
\legend{}
}%Plot
\hfill
\PlotLarge[ylabel=]{wikipedia}{%
%% MULTIPLOT(m.algo) $wplot('wikipedia')
%% CONFIG file=plot/w_all_wikipedia.tex type=tex
\input{plot/w_all_wikipedia.tex}
\legend{}
}%Plot


%% UNDEF wplot(file)
\ref{legPlotWelchAll}

\caption{Juxtaposition of all \LZTrie{} implementations for the \LZW{} factorization.
}
\label{figPlotWelchAll}
\end{figure}

\begin{figure}[t!]
%% DEFINE wplot(file)
%% SELECT m.mem*8.0/m.n as x, MIN(t.time*1e6/t.n) as y, MULTIPLOT from tudocomp_memory AS m 
%% JOIN tudocomp_time AS t ON m.algo = t.algo AND t.type = m.type AND t.file = m.file AND t.action = m.action
%% WHERE m.type='w' AND m.file=$file AND m.action='compression' 
%% AND m.algo in ('hash', 'hashplus', 'rolling', 'rollingplus', 'binary', 'ternary', 'judy')
%% GROUP BY MULTIPLOT,x ORDER BY MULTIPLOT,x

\pgfplotsset{squeezedPlot,
	xlabel style={align=center,yshift=-1em},
	xlabel={memory\\ $[$bits/$n]$},
	xticklabel style={rotate=90, anchor=east},
	ticklabel style={font=\small},
	width=\linewidth,
	height=\linewidth,
	y tick label style={%
		/pgf/number format/.cd,
		fixed relative,
		precision=2,
		zerofill,
		/tikz/.cd,
	},
}


\PlotSmall[bitsXaxis,ylabel={time [$\mu\text{s}/n$]},
xlabel=,
appendLegend,
legend to name={legPlotWelchHighMem},
legend columns=5]{dna}{%

%% MULTIPLOT(m.algo) $wplot('dna')
%% CONFIG file=plot/w_highmem_dna.tex type=tex
\input{plot/w_highmem_dna.tex}
}%Plot
%
\PlotSmall[bitsXaxis,xlabel=]{english}{%
%% MULTIPLOT(m.algo) $wplot('english')
%% CONFIG file=plot/w_highmem_english.tex type=tex
\input{plot/w_highmem_english.tex}
\legend{}
}%Plot
%
\PlotSmall[bitsXaxis,xlabel=]{proteins}{%
%% MULTIPLOT(m.algo) $wplot('proteins')
%% CONFIG file=plot/w_highmem_proteins.tex type=tex
\input{plot/w_highmem_proteins.tex}
\legend{}
}%Plot
%
\PlotSmall[bitsXaxis,xlabel=]{xml}{%
%% MULTIPLOT(m.algo) $wplot('xml')
%% CONFIG file=plot/w_highmem_xml.tex type=tex
\input{plot/w_highmem_xml.tex}
\legend{}
}%Plot

\PlotSmall[bitsXaxis,ylabel={time [$\mu\text{s}/n$]}]{commoncrawl}{%
%% MULTIPLOT(m.algo) $wplot('commoncrawl')
%% CONFIG file=plot/w_highmem_commoncrawl.tex type=tex
\input{plot/w_highmem_commoncrawl.tex}
\legend{}
}%Plot
%
\PlotSmall{fibonacci}{%
%% MULTIPLOT(m.algo) $wplot('fibonacci')
%% CONFIG file=plot/w_highmem_fibonacci.tex type=tex
\input{plot/w_highmem_fibonacci.tex}
\legend{}
}%Plot
%
\PlotSmall[bitsXaxis]{gutenberg}{%
%% MULTIPLOT(m.algo) $wplot('gutenberg')
%% CONFIG file=plot/w_highmem_gutenberg.tex type=tex
\input{plot/w_highmem_gutenberg.tex}
\legend{}
}%Plot
%
\PlotSmall[bitsXaxis]{wikipedia}{%
%% MULTIPLOT(m.algo) $wplot('wikipedia')
%% CONFIG file=plot/w_highmem_wikipedia.tex type=tex
\input{plot/w_highmem_wikipedia.tex}
\legend{}
}%Plot

%% UNDEF wplot(file)
\ref{legPlotWelchHighMem}

%% DEFINE wplot(file)
%% SELECT m.mem*8.0/m.n as x, MIN(t.time*1e6/t.n) as y, MULTIPLOT from tudocomp_memory AS m 
%% JOIN tudocomp_time AS t ON m.algo = t.algo AND t.type = m.type AND t.file = m.file AND t.action = m.action
%% WHERE m.type='w' AND m.file=$file AND m.action='compression' 
%% AND m.algo not in ('hash', 'hashplus', 'rolling', 'rollingplus', 'rolling128', 'rolling128plus', 'binary', 'ternary', 'judy', 'binaryk', 'binaryMTF', 'binarykP', 'chain0P', 'compact_kvP', 'binarysorted', 'exthash')
%% GROUP BY MULTIPLOT,x ORDER BY MULTIPLOT,x

\PlotSmall[bitsXaxis,ylabel={time [$\mu\text{s}/n$]},
appendLegend,
xlabel=,
legend to name={legPlotWelch},
legend columns=5]{dna}{%
%% MULTIPLOT(m.algo) $wplot('dna')
%% CONFIG file=plot/w_dna.tex type=tex
\input{plot/w_dna.tex}
}%Plot
%
\PlotSmall[bitsXaxis,xlabel=]{english}{%
%% MULTIPLOT(m.algo) $wplot('english')
%% CONFIG file=plot/w_english.tex type=tex
\input{plot/w_english.tex}
\legend{}
}%Plot
%
\PlotSmall[bitsXaxis,xlabel=]{proteins}{%
%% MULTIPLOT(m.algo) $wplot('proteins')
%% CONFIG file=plot/w_proteins.tex type=tex
\input{plot/w_proteins.tex}
\legend{}
}%Plot
%
\PlotSmall[bitsXaxis,xlabel=]{xml}{%
%% MULTIPLOT(m.algo) $wplot('xml')
%% CONFIG file=plot/w_xml.tex type=tex
\input{plot/w_xml.tex}
\legend{}
}%Plot

\PlotSmall[bitsXaxis,ylabel={time [$\mu\text{s}/n$]}]{commoncrawl}{%
%% MULTIPLOT(m.algo) $wplot('commoncrawl')
%% CONFIG file=plot/w_commoncrawl.tex type=tex
\input{plot/w_commoncrawl.tex}
\legend{}
}%Plot
%
\PlotSmall[tinyX]{fibonacci}{%
%% MULTIPLOT(m.algo) $wplot('fibonacci')
%% CONFIG file=plot/w_fibonacci.tex type=tex
\input{plot/w_fibonacci.tex}
\legend{}
}%Plot
%
\PlotSmall[bitsXaxis]{gutenberg}{%
%% MULTIPLOT(m.algo) $wplot('gutenberg')
%% CONFIG file=plot/w_gutenberg.tex type=tex
\input{plot/w_gutenberg.tex}
\legend{}
}%Plot
%
\PlotSmall[bitsXaxis]{wikipedia}{%
%% MULTIPLOT(m.algo) $wplot('wikipedia')
%% CONFIG file=plot/w_wikipedia.tex type=tex
\input{plot/w_wikipedia.tex}
\legend{}
}%Plot

\ref{legPlotWelch}

\caption{Evaluation of our \LZTrie{} implementations for the \LZW{} factorization.
}
\label{figPlotWelchDivided}
\end{figure}


\begin{table}[t!]
%% DEFINE subquery(file)
%% SELECT m.algo AS algo, printf("%.2f", m.mem*8.0/m.n) AS mem, printf("%.2f", MIN(t.time*1e6/t.n)) AS time, m.type AS type
%% FROM tudocomp_memory AS m 
%% JOIN tudocomp_time AS t ON m.algo = t.algo AND t.type = m.type AND t.file = m.file AND t.action = m.action
%% WHERE m.file=$file AND m.action='compression' 
%% AND m.algo in ('rolling', 'rollingplus', 'rolling128', 'rolling128plus')
%% GROUP BY m.algo, m.type 

%% DEFINE rolling(file)
%% SELECT
%% "JUSTIFY(" || eight.algo || ")", eight.time, eight.mem, w.time, w.mem
%% FROM ($subquery($file)) AS eight
%% INNER JOIN ($subquery($file)) as w ON eight.algo = w.algo
%% WHERE eight.type = '78' and w.type = 'w'
%% ORDER BY eight.algo = 'rolling' DESC, eight.algo = 'rollingplus' DESC, eight.algo = 'rolling128' DESC, eight.algo = 'rolling128plus' DESC


		\setlength{\tabcolsep}{0.4em}

		\begin{tabular}{l*{4}{r}}
			\toprule
			&	\multicolumn{2}{c}{\LZEight{}} & \multicolumn{2}{c}{\LZW{}}
			\\ \cmidrule(lr){2-3} \cmidrule(lr){4-5}
	dataset / & time & space & time & space \\
	trie			& [$\mu \text{s}/n$] & $\frac{\text{bits}}{n}$ & [$\mu \text{s}/n$] & $\frac{\text{bits}}{n}$ \\
\multicolumn{4}{l}{\textsc{dna}}
\\\midrule
%% TABULAR $rolling('dna')
%% CONFIG file=plot/tabRollingDna.tex
%
%%
\csname @@input\endcsname {plot/tabRollingDna.tex}
%
\multicolumn{4}{l}{\textsc{english}}%
\\\midrule
%% TABULAR $rolling('english')
%% CONFIG file=plot/tabRollingEnglish.tex
%
%%
\csname @@input\endcsname {plot/tabRollingEnglish.tex}
%
\multicolumn{4}{l}{\textsc{proteins}}%
\\\midrule
%% TABULAR $rolling('proteins')
%% CONFIG file=plot/tabRollingProteins.tex
%
%%
\csname @@input\endcsname {plot/tabRollingProteins.tex}
%
\multicolumn{4}{l}{\textsc{xml}} 
\\\midrule
%% TABULAR $rolling('xml')
%% CONFIG file=plot/tabRollingXml.tex
%
%%
\csname @@input\endcsname {plot/tabRollingXml.tex}
 			 \bottomrule
   \end{tabular}
		\begin{tabular}{l*{4}{r}}
			\toprule
			&	\multicolumn{2}{c}{\LZEight{}} & \multicolumn{2}{c}{\LZW{}}
			\\ \cmidrule(lr){2-3} \cmidrule(lr){4-5}
	dataset / & time & space & time & space \\
	trie			& [$\mu \text{s}/n$] & $\frac{\text{bits}}{n}$ & [$\mu \text{s}/n$] & $\frac{\text{bits}}{n}$ \\
%
\multicolumn{4}{l}{\textsc{commoncrawl}} 
\\\midrule
%% TABULAR $rolling('commoncrawl')
%% CONFIG file=plot/tabRollingcommoncrawl.tex
%
%%
\csname @@input\endcsname {plot/tabRollingcommoncrawl.tex}
%
\multicolumn{4}{l}{\textsc{fibonacci}} 
\\\midrule
%% TABULAR $rolling('fibonacci')
%% CONFIG file=plot/tabRollingfibonacci.tex
%
%%
\csname @@input\endcsname {plot/tabRollingfibonacci.tex}
%
\multicolumn{4}{l}{\textsc{gutenberg}} 
\\\midrule
%% TABULAR $rolling('gutenberg')
%% CONFIG file=plot/tabRollinggutenberg.tex
%
%%
\csname @@input\endcsname {plot/tabRollinggutenberg.tex}
%
\multicolumn{4}{l}{\textsc{wikipedia}} 
\\\midrule
%% TABULAR $rolling('wikipedia')
%% CONFIG file=plot/tabRollingwikipedia.tex
%
%%
\csname @@input\endcsname {plot/tabRollingwikipedia.tex}
 			 \bottomrule
   \end{tabular}
   \caption{Performance comparison of 64-bit and 128-bit fingerprints of \iRolling{} with its variant \iRolling{+} when computing the \LZEight{} and \LZW{} factorization.}
\label{tableLargeBitsRolling}
%%UNDEF subquery
%%UNDEF rolling
   \end{table}

\begin{figure}[t!]
	\centering{%

		\pgfplotsset{%
			myaxisstyle/.style={%
	ybar=8pt,
	width=\linewidth,
    enlarge x limits=0.1,
	legend columns=5,
    ylabel={compression ratio},
	symbolic x coords={dna, english,xml,proteins,fibonacci,gutenberg,wikipedia,commoncrawl},
    xtick=data,
	ymin=0,
	ymax=80,
	y=2pt,
	x tick label style={font=\scriptsize\scshape},
    nodes near coords,
	nodes near coords style={font=\normalfont,rotate=90,anchor=west},
	bar width = 0pt,
cycle multi list={%
		Dark2-6\nextlist
		dashed,solid\nextlist
}}
}


\pgfplotsset{cycle list/Dark2-6}
\begin{tikzpicture}
\begin{axis}[myaxisstyle,
	legend to name={legCompressionRatio},
	x tick label style={rotate=90,anchor=east}
	]
%% MULTIPLOT(type) 
%% SELECT file AS x, compressedsize*100.0/n AS y, MULTIPLOT
%% FROM tudocomp_memory
%% WHERE action = 'compression'
%% AND file NOT IN ('english.1')
%% GROUP BY MULTIPLOT,x ORDER BY MULTIPLOT,x
%% CONFIG file=plot/compratio2.tex colorcache=none 

%% MULTIPLOT(program) 
%% SELECT file AS x, compressedsize*100.0/n AS y, MULTIPLOT
%% FROM unixcompress 
%% WHERE action = 'compression'
%% AND file NOT IN ('english.1')
%% GROUP BY MULTIPLOT,x ORDER BY MULTIPLOT,x
%% CONFIG file=plot/compratio2.tex colorcache=none mode=a

% MULTIPLOT(algo) 
% SELECT s.file AS x, SUM(s.compressedsize)*100.0/t.n AS y, MULTIPLOT
% FROM splitrun AS s 
% INNER JOIN (SELECT t.file, t.n FROM tudocomp_memory AS t where t.action = 'compression' group by t.file) AS t 
% ON t.file = s.file 
% WHERE s.action = 'compression' 
% AND s.file NOT IN ('english.1')
% GROUP BY MULTIPLOT,x ORDER BY MULTIPLOT,x
% CONFIG file=plot/compratio2.tex colorcache=none mode=a

\input{plot/compratio2.tex}
\end{axis}
\end{tikzpicture}

	}%centering
\ref{legCompressionRatio}
\caption{Compression ratios of the classic coding of \LZEight{} and \LZW{}
	compared with the Unix tool \texttt{compress} with its best compression.
% and with \texttt{split}, which is the most memory-efficient LZ78 factorization algorithm that has to discard its dictionary when its memory reaches the memory limit of the most-efficient LZ78 algorithm computing the Bonsai coding.
% We have no data points for \texttt{split} and \textsc{fibonacci} since one of the algorithms using the classic coding use less memory than all Bonsai-coding algorithms.
}
	\label{figCompressionRatio}
\end{figure}


\begin{figure}[t!]
\pgfplotsset{squeezedPlot,bitsXaxis,
	xlabel={memory $[$bits/$n]$},
	ylabel={time [$\mu\text{s}/n$]},
	width=\linewidth,
	height=0.6\linewidth,
}

%% DEFINE grpsubquery(file)
%% SELECT m.mem*8.0/m.n AS x, MIN(t.time*1e6/t.n) AS y, 'chainbeta' AS name from tudocomp_memory AS m 
%% JOIN tudocomp_time AS t ON m.algo = t.algo AND t.type = m.type AND t.file = m.file AND t.action = m.action
%% WHERE m.type='78' AND m.file=$file AND m.action='compression' 
%% AND m.algo IN ('chain0', 'chain10', 'chain20', 'chain30') 
%% GROUP BY m.algo


%% DEFINE GrpCompMemTime(file)
%% SELECT
%% m.x AS x, m.y AS y, MULTIPLOT
%% FROM ($grpsubquery($file)) AS m


%% DEFINE lowCompMemTime(file,clause)
%% SELECT (mem*8.0)/n AS x, MIN((time*1e6)/n) as y, MULTIPLOT FROM "low" 
%% WHERE "action" = 'compression'
%% AND "file" = $file AND "tabletype" in ('-1','2')
%% AND indextype NOT IN ('6') $clause
%% AND useEliasDisplacement='1'
%% GROUP BY MULTIPLOT,factor ORDER BY MULTIPLOT,x

%% DEFINE 78CompMemTime(file)
%% SELECT m.mem*8.0/m.n as x, MIN(t.time*1e6/t.n) as y, MULTIPLOT from tudocomp_memory AS m 
%% JOIN tudocomp_time AS t ON m.algo = t.algo AND t.type = m.type AND t.file = m.file AND t.action = m.action
%% WHERE m.type='78' AND m.file=$file AND m.action='compression' 
%% AND m.algo IN ('binary', 'ternary')
%% GROUP BY MULTIPLOT,x ORDER BY MULTIPLOT,x


\PlotLarge[xlabel=,
appendLegend,
legend to name={legPlotLowCompMemTime},
legend columns=5]
{dna}{%
%% MULTIPLOT(indextype) $lowCompMemTime('dna',)
%% CONFIG file=plot/low_dna.tex

%% MULTIPLOT(m.algo) $78CompMemTime('dna')
%% CONFIG file=plot/low_dna.tex mode=a

%% MULTIPLOT(name) $GrpCompMemTime('dna')
%% CONFIG file=plot/low_dna.tex mode=a

\input{plot/low_dna.tex}
}%plot
%
\PlotLarge[xlabel=,ylabel=]{english}{%
%% MULTIPLOT(indextype) $lowCompMemTime('english',)
%% CONFIG file=plot/low_english.tex

%% MULTIPLOT(m.algo) $78CompMemTime('english')
%% CONFIG file=plot/low_english.tex mode=a

%% MULTIPLOT(name) $GrpCompMemTime('english')
%% CONFIG file=plot/low_english.tex mode=a

\input{plot/low_english.tex}
\legend{}
}%plot

\PlotLarge[xlabel=]{proteins}{%
%% MULTIPLOT(indextype) $lowCompMemTime('proteins',)
%% CONFIG file=plot/low_proteins.tex

%% MULTIPLOT(m.algo) $78CompMemTime('proteins')
%% CONFIG file=plot/low_proteins.tex mode=a

%% MULTIPLOT(name) $GrpCompMemTime('proteins')
%% CONFIG file=plot/low_proteins.tex mode=a

\input{plot/low_proteins.tex}
\legend{}
}%plot
%
\PlotLarge[xlabel=,ylabel=]{xml}{%
%% MULTIPLOT(indextype) $lowCompMemTime('xml',)
%% CONFIG file=plot/low_xml.tex

%% MULTIPLOT(m.algo) $78CompMemTime('xml')
%% CONFIG file=plot/low_xml.tex mode=a

%% MULTIPLOT(name) $GrpCompMemTime('xml')
%% CONFIG file=plot/low_xml.tex mode=a

\input{plot/low_xml.tex}
\legend{}
}%plot

\PlotLarge[xlabel=]{commoncrawl}{%
%% MULTIPLOT(indextype) $lowCompMemTime('commoncrawl',)
%% CONFIG file=plot/low_commoncrawl.tex

%% MULTIPLOT(m.algo) $78CompMemTime('commoncrawl')
%% CONFIG file=plot/low_commoncrawl.tex mode=a

%% MULTIPLOT(name) $GrpCompMemTime('commoncrawl')
%% CONFIG file=plot/low_commoncrawl.tex mode=a

\input{plot/low_commoncrawl.tex}
\legend{}
}%plot
%
\PlotLarge[xlabel=,ylabel=]{fibonacci}{%
%% MULTIPLOT(indextype) $lowCompMemTime('fibonacci', AND indextype != '0' AND indextype != '1')
%% CONFIG file=plot/low_fibonacci.tex

%% MULTIPLOT(m.algo) $78CompMemTime('fibonacci')
%% CONFIG file=plot/low_fibonacci.tex mode=a

%% MULTIPLOT(name) $GrpCompMemTime('fibonacci')
%% CONFIG file=plot/low_fibonacci.tex mode=a

\input{plot/low_fibonacci.tex}
\legend{}
}%plot

\PlotLarge{gutenberg}{%
%% MULTIPLOT(indextype) $lowCompMemTime('gutenberg',)
%% CONFIG file=plot/low_gutenberg.tex

%% MULTIPLOT(m.algo) $78CompMemTime('gutenberg')
%% CONFIG file=plot/low_gutenberg.tex mode=a

%% MULTIPLOT(name) $GrpCompMemTime('gutenberg')
%% CONFIG file=plot/low_gutenberg.tex mode=a

\input{plot/low_gutenberg.tex}
\legend{}
}%plot
%
\PlotLarge[ylabel=]{wikipedia}{%
%% MULTIPLOT(indextype) $lowCompMemTime('wikipedia',)
%% CONFIG file=plot/low_wikipedia.tex

%% MULTIPLOT(m.algo) $78CompMemTime('wikipedia')
%% CONFIG file=plot/low_wikipedia.tex mode=a

%% MULTIPLOT(name) $GrpCompMemTime('wikipedia')
%% CONFIG file=plot/low_wikipedia.tex mode=a

\input{plot/low_wikipedia.tex}
\legend{}
}%plot


%% UNDEF lowCompMemTime
%% UNDEF 78CompMemTime

\ref{legPlotLowCompMemTime}

\caption{Maximum RAM and time used during the \LZEight{} factorizations with the Bonsai coding variants and the space-efficient variants.}
	\label{figPlotLowCompMemTime}
\end{figure}



\begin{figure}[t!]
\pgfplotsset{squeezedPlot,bitsXaxis,
	xlabel={memory $[$bits/$n]$},
	ylabel={time [$\mu\text{s}/n$]},
	width=\linewidth,
	height=0.6\linewidth,
}

%% DEFINE lowDecMemTime(file,clause)
%% SELECT (mem*8.0)/n AS x, MIN((time*1e6)/n) as y, MULTIPLOT FROM "low" 
%% WHERE "action" = 'decompression'
%% AND "file" = $file AND "tabletype" in ('-1','2')
%% AND indextype NOT IN ('6') $clause
%% AND useEliasDisplacement='1'
%% GROUP BY MULTIPLOT,factor ORDER BY MULTIPLOT,x

%% DEFINE 78DecMemTime(file)
%% SELECT MIN(m.mem*8.0/m.n) as x, MIN(t.time*1e6/t.n) as y, MULTIPLOT from tudocomp_memory AS m 
%% JOIN tudocomp_time AS t ON m.algo = t.algo AND t.type = m.type AND t.file = m.file AND t.action = m.action
%% WHERE m.file=$file AND m.action='decompression' AND m.type = '78'
%% AND m.algo NOT IN ('rolling', 'rollingplus', 'rolling128', 'rolling128plus', 'exthash', 'hash', 'hashplus', 'judy' )
%% GROUP BY MULTIPLOT ORDER BY x

\PlotLarge[xlabel=,
appendLegend,
legend to name={legPlotLowDecMemTime},
legend columns=5]
{dna}{%
%% MULTIPLOT(indextype) $lowDecMemTime('dna',)
%% CONFIG file=plot/low_dec_dna.tex

%% MULTIPLOT(m.type) $78DecMemTime('dna')
%% CONFIG file=plot/low_dec_dna.tex mode=a

\input{plot/low_dec_dna.tex}
}%plot
%
\PlotLarge[xlabel=,ylabel=]{english}{%
%% MULTIPLOT(indextype) $lowDecMemTime('english',)
%% CONFIG file=plot/low_dec_english.tex

%% MULTIPLOT(m.type) $78DecMemTime('english')
%% CONFIG file=plot/low_dec_english.tex mode=a

\input{plot/low_dec_english.tex}
\legend{}
}%plot

\PlotLarge[xlabel=]{proteins}{%
%% MULTIPLOT(indextype) $lowDecMemTime('proteins',)
%% CONFIG file=plot/low_dec_proteins.tex

%% MULTIPLOT(m.type) $78DecMemTime('proteins')
%% CONFIG file=plot/low_dec_proteins.tex mode=a

\input{plot/low_dec_proteins.tex}
\legend{}
}%plot
%
\PlotLarge[xlabel=,ylabel=]{xml}{%
%% MULTIPLOT(indextype) $lowDecMemTime('xml',)
%% CONFIG file=plot/low_dec_xml.tex

%% MULTIPLOT(m.type) $78DecMemTime('xml')
%% CONFIG file=plot/low_dec_xml.tex mode=a

\input{plot/low_dec_xml.tex}
\legend{}
}%plot
%

\PlotLarge[xlabel=]{commoncrawl}{%
%% MULTIPLOT(indextype) $lowDecMemTime('commoncrawl',)
%% CONFIG file=plot/low_dec_commoncrawl.tex

%% MULTIPLOT(m.type) $78DecMemTime('commoncrawl')
%% CONFIG file=plot/low_dec_commoncrawl.tex mode=a

\input{plot/low_dec_commoncrawl.tex}
\legend{}
}%plot
%
\PlotLarge[xlabel=,ylabel=,x tick label style={%rotate=90,anchor=east,
        /pgf/number format/.cd,
            fixed,
            fixed zerofill,
            precision=2,
        /tikz/.cd
    }
]{fibonacci}{%
%% MULTIPLOT(indextype) $lowDecMemTime('fibonacci', AND indextype != '0' AND indextype != '1')
%% CONFIG file=plot/low_dec_fibonacci.tex

%% MULTIPLOT(m.type) $78DecMemTime('fibonacci')
%% CONFIG file=plot/low_dec_fibonacci.tex mode=a

\input{plot/low_dec_fibonacci.tex}
\legend{}
}%plot

\PlotLarge{gutenberg}{%
%% MULTIPLOT(indextype) $lowDecMemTime('gutenberg',)
%% CONFIG file=plot/low_dec_gutenberg.tex

%% MULTIPLOT(m.type) $78DecMemTime('gutenberg')
%% CONFIG file=plot/low_dec_gutenberg.tex mode=a

\input{plot/low_dec_gutenberg.tex}
\legend{}
}%plot
%
\PlotLarge[ylabel=]{wikipedia}{%
%% MULTIPLOT(indextype) $lowDecMemTime('wikipedia',)
%% CONFIG file=plot/low_dec_wikipedia.tex

%% MULTIPLOT(m.type) $78DecMemTime('wikipedia')
%% CONFIG file=plot/low_dec_wikipedia.tex mode=a

\input{plot/low_dec_wikipedia.tex}
\legend{}
}%plot
%

%% UNDEF lowDecMemTime
%% UNDEF 78DecMemTime

\ref{legPlotLowDecMemTime}

\caption{Maximum RAM and time used during decoding of a file stored with one of the Bonsai coding variants.
This is compared with the decompression of the same file compressed with \LZEight{} using the classic coding. }
\label{figPlotLowDecMemTime}
\end{figure}

\begin{figure}[t!]
	\centering{%

% SELECT low.indextype, printf("%.2f", 1+low.factor/100.0), printf("%.2f", MIN(low.compressedsize*100.0/low.n)), printf("%.2f", MIN(g.compressedsize*100.0/low.n))
% FROM low 
% JOIN low AS g ON g.action = low.action AND g.indextype = low.indextype AND g.file = low.file AND g.factor = low.factor AND g.useEliasDisplacement='1'
% WHERE low.action = 'compression'
% AND low.file = 'fibonacci'
% GROUP BY low.factor,low.indextype,low.file order by low.file,low.indextype;
%
% AND low.useEliasDisplacement='0'
% AND low."tabletype" in ('-1','2')
% AND low.indextype NOT IN ('6','8','9')
% AND low.indextype = '0'

%% DEFINE tableCompSize(file, indextype)
%% SELECT printf("%.2f", 1+low.factor/100.0), printf("%.2f", MIN(low.compressedsize*100.0/low.n)), printf("%.2f", MIN(g.compressedsize*100.0/low.n))
%% FROM low 
%% JOIN low AS g ON g.action = low.action AND g.indextype = low.indextype AND g.file = low.file AND g.factor = low.factor AND g.useEliasDisplacement='1'
%% WHERE low.action = 'compression' AND low.useEliasDisplacement='0'
%% AND low."tabletype" in ('-1','2')
%% AND low.indextype NOT IN ('6','8','9')
%% AND low.file = $file
%% AND low.indextype = $indextype
%% GROUP BY low.factor,low.indextype,low.file order by low.file,low.indextype


\begin{tabular}{*{3}{r}}
	\multicolumn{3}{c}{\Dataset{english}, \iGrow{}}\\
\toprule
 & \multicolumn{2}{c}{Ratio}\\
			\cmidrule(lr){2-3}
$\frac{1}{\alpha}$ & plain & $\gamma$
\\\midrule
%% TABULAR 
%% $tableCompSize('english','4')
%% CONFIG file=plot/tabCompSizeEnglish4.tex
\csname @@input\endcsname {plot/tabCompSizeEnglish4.tex}
%
\bottomrule
		\end{tabular}
\begin{tabular}{*{3}{r}}
	\multicolumn{3}{c}{\Dataset{proteins}, \iFix{3}}\\
\toprule
 & \multicolumn{2}{c}{Ratio}\\
			\cmidrule(lr){2-3}
$\frac{1}{\alpha}$ & plain & $\gamma$
\\\midrule
%% TABULAR 
%% $tableCompSize('proteins','1')
%% CONFIG file=plot/tabCompSizeProteins1.tex
\csname @@input\endcsname {plot/tabCompSizeProteins1.tex}
%
\bottomrule
		\end{tabular}
\begin{tabular}{*{3}{r}}
	\multicolumn{3}{c}{\Dataset{fibonacci}, \iFix{}}\\
\toprule
 & \multicolumn{2}{c}{Ratio}\\
			\cmidrule(lr){2-3}
$\frac{1}{\alpha}$ & plain & $\gamma$
\\\midrule
%% TABULAR 
%% $tableCompSize('fibonacci','0')
%% CONFIG file=plot/tabCompSizeFibonacci0.tex
\csname @@input\endcsname {plot/tabCompSizeFibonacci0.tex}
%
\bottomrule
		\end{tabular}

		\pgfplotsset{%
			myaxisstyle/.style={%
	y=2pt,
	ybar=7pt,
	width=\linewidth,
	enlarge x limits=0.1,
	legend columns=5
    ylabel={compression ratio},
	symbolic x coords={dna, english,xml,proteins,fibonacci,gutenberg,wikipedia,commoncrawl},
    xtick=data,
	ymin=0,
	ymax=100,
	x tick label style={rotate=45,anchor=east,font=\scshape},
    nodes near coords,
	bar width = 0pt,
	nodes near coords style={font=\small,rotate=90,anchor=west},
cycle multi list={%
		Dark2-6\nextlist
		dashed,solid\nextlist
}}
}

\pgfplotsset{cycle list/Dark2-6}
  \begin{tikzpicture}
\begin{axis}[myaxisstyle,
	legend to name={legPlotLowCompressionRatioGamma},
	]

%% MULTIPLOT(low.indextype)
%% SELECT low.file AS x, MIN(low.compressedsize)*100.0/low.n AS y, MULTIPLOT
%% FROM low 
%% WHERE low.action = 'compression'
%% AND low."tabletype" in ('-1','2')
%% AND low.indextype IN ('0','1')
%% AND file NOT IN ('english.1')
%% AND useEliasDisplacement='1'
%% GROUP BY MULTIPLOT,x ORDER BY MULTIPLOT,x
%% CONFIG file=plot/low_compratiogamma.tex colorcache=none

%% MULTIPLOT(low.indextype)
%% SELECT low.file AS x, MIN(low.compressedsize)*100.0/low.n AS y, MULTIPLOT
%% FROM 
%% (
%% SELECT file, compressedsize, n, indextype || 'gamma' AS indextype, indextype, action, file FROM low WHERE useEliasDisplacement='1'
%% ) AS low
%% WHERE action = 'compression'
%% AND "indextype" IN ('0gamma','1gamma')
%% AND "file" NOT IN ('english.1')
%% GROUP BY MULTIPLOT,x ORDER BY MULTIPLOT,x
%% CONFIG file=plot/low_compratiogamma.tex colorcache=none mode=a

\input{plot/low_compratiogamma.tex}
\end{axis}
\end{tikzpicture}

	}%centering
\ref{legPlotLowCompressionRatioGamma}
\caption{Top: Compression ratios of the plain Bonsai coding and of the Bonsai coding with the stored displacement values of the first layer in \iLayered{2} or \iLayered{3} encoded with Elias-$\gamma$ (labeled with $\bsq{\gamma}$).
	Bottom: Compression ratios of \iFix{} (resp.\ \iFix{3}) with and without Elias-$\gamma$ encoding.
	We add $\gamma$ in the subscript if we apply Elias-$\gamma$ encoding.
	For each dataset, we selected the best compression ratios while varying the maximum load factor of the Bonsai table.
}
\label{figPlotLowCompressionRatioGamma}
\end{figure}

\begin{figure}[t!]
	\centering{%

		\pgfplotsset{%
			myaxisstyle/.style={%
	y=2pt,
	ybar=8pt,
	width=\linewidth,
	enlarge x limits=0.2,
	legend columns=5
    ylabel={compression ratio},
	symbolic x coords={dna, english,xml,proteins,fibonacci,gutenberg,wikipedia,commoncrawl},
    xtick=data,
	ymin=0,
	ymax=100,
	x tick label style={font=\scshape},
    nodes near coords,
	nodes near coords style={font=\normalsize,rotate=90,anchor=west},
	bar width = 0pt,
cycle multi list={%
		solid,loosely dashed\nextlist
		Dark2-6\nextlist
}}
}



\pgfplotsset{cycle list/Dark2-6}
  \begin{tikzpicture}
\begin{axis}[myaxisstyle,
	legend to name={legLZLowCompressionRatio},
legend columns=8,
	]
%% MULTIPLOT(low.indextype)
%% SELECT low.file AS x, MIN(low.compressedsize)*100.0/low.n AS y, MULTIPLOT
%% FROM low 
%% WHERE low.action = 'compression'
%% AND low."tabletype" in ('-1','2')
%% AND low.indextype NOT IN ('6','8','9')
%% AND low.file IN ('xml', 'dna', 'english', 'proteins')
%% AND useEliasDisplacement='1'
%% GROUP BY MULTIPLOT,x ORDER BY MULTIPLOT,x
%% CONFIG file=plot/low_compratio.tex colorcache=none

%% MULTIPLOT(type) 
%% SELECT file AS x, compressedsize*100.0/n AS y, MULTIPLOT
%% FROM tudocomp_memory
%% WHERE action = 'compression'
%% AND file IN ('xml', 'dna', 'english', 'proteins')
%% AND type = '78'
%% GROUP BY MULTIPLOT,x ORDER BY MULTIPLOT,x
%% CONFIG file=plot/low_compratio.tex colorcache=none mode=a

% MULTIPLOT(program) 
% SELECT file AS x, compressedsize*100.0/n AS y, MULTIPLOT
% FROM unixcompress 
% WHERE action = 'compression'
% AND file IN ('xml', 'dna', 'english', 'proteins')
% GROUP BY MULTIPLOT,x ORDER BY MULTIPLOT,x
% CONFIG file=plot/low_compratio.tex colorcache=none mode=a

% MULTIPLOT(algo) 
% SELECT s.file AS x, SUM(s.compressedsize)*100.0/t.n AS y, MULTIPLOT
% FROM splitrun AS s 
% INNER JOIN (SELECT t.file, t.n FROM tudocomp_memory AS t where t.action = 'compression' group by t.file) AS t 
% ON t.file = s.file 
% WHERE s.action = 'compression' 
% AND s.file IN ('xml', 'dna', 'english', 'proteins')
% GROUP BY MULTIPLOT,x ORDER BY MULTIPLOT,x
% CONFIG file=plot/low_compratio.tex colorcache=none mode=a

\input{plot/low_compratio.tex}
\end{axis}
\end{tikzpicture}
\begin{tikzpicture}
\begin{axis}[myaxisstyle]
%% MULTIPLOT(low.indextype)
%% SELECT low.file AS x, MIN(low.compressedsize)*100.0/low.n AS y, MULTIPLOT
%% FROM low 
%% WHERE low.action = 'compression'
%% AND low."tabletype" in ('-1','2')
%% AND low.indextype NOT IN ('6','8','9')
%% AND low.file NOT IN ('english.1', 'xml', 'dna', 'english', 'proteins')
%% AND useEliasDisplacement='1'
%% GROUP BY MULTIPLOT,x ORDER BY MULTIPLOT,x
%% CONFIG file=plot/low_compratio2.tex colorcache=none

%% MULTIPLOT(type) 
%% SELECT file AS x, compressedsize*100.0/n AS y, MULTIPLOT
%% FROM tudocomp_memory
%% WHERE action = 'compression'
%% AND file NOT IN ('english.1', 'xml', 'dna', 'english', 'proteins')
%% AND type = '78'
%% GROUP BY MULTIPLOT,x ORDER BY MULTIPLOT,x
%% CONFIG file=plot/low_compratio2.tex colorcache=none mode=a

% MULTIPLOT(program) 
% SELECT file AS x, compressedsize*100.0/n AS y, MULTIPLOT
% FROM unixcompress 
% WHERE action = 'compression'
% AND file NOT IN ('english.1', 'xml', 'dna', 'english', 'proteins')
% GROUP BY MULTIPLOT,x ORDER BY MULTIPLOT,x
% CONFIG file=plot/low_compratio2.tex colorcache=none mode=a

% MULTIPLOT(algo) 
% SELECT s.file AS x, SUM(s.compressedsize)*100.0/t.n AS y, MULTIPLOT
% FROM splitrun AS s 
% INNER JOIN (SELECT t.file, t.n FROM tudocomp_memory AS t where t.action = 'compression' group by t.file) AS t 
% ON t.file = s.file 
% WHERE s.action = 'compression' 
% AND s.file NOT IN ('english.1', 'xml', 'dna', 'english', 'proteins')
% GROUP BY MULTIPLOT,x ORDER BY MULTIPLOT,x
% CONFIG file=plot/low_compratio2.tex colorcache=none mode=a

\input{plot/low_compratio2.tex}
\legend{}
\end{axis}
\end{tikzpicture}

	}%centering
\ref{legLZLowCompressionRatio}
\caption{Compression ratios of our algorithms computing the Bonsai coding. 
We omit the $\bsq{\textup{s}}$ variants of \iGrow{} and \iGrow{3} since they only make a difference in the compression speed, not in the final output. 
We compare these ratios with those of the classic coding of \LZEight{}.
For each dataset, we selected the best compression ratios while varying the maximum load factor of the Bonsai table.}

\label{figLZLowCompressionRatio}
\end{figure}


\begin{figure}[t!]
\pgfplotsset{squeezedPlot,bitsXaxis,
	ylabel={time [$\mu\text{s}/n$]},
	xlabel={compression overhead},
	xticklabel={\pgfmathparse{\tick*100}\pgfmathprintnumber{\pgfmathresult}\%},
	width=\linewidth,
	height=0.6\linewidth,
	xmax=1,
}

%% DEFINE lowCompTimeOutputSize(file, clause)
%% SELECT (low.compressedsize*1.0/m.compressedsize-1) AS x, MIN(low.time*1e6/low.n) AS y, MULTIPLOT
%% FROM low 
%% JOIN (SELECT compressedsize, file FROM "tudocomp_memory" where action = 'compression' AND type = '78' group by file) AS m 
%% ON m.file = low.file 
%% WHERE low.action = 'compression' AND low.file = $file $clause
%% AND low."tabletype" in ('-1','2')
%% AND low.indextype NOT IN ('6')
%% AND low.useEliasDisplacement='1'
%% GROUP BY MULTIPLOT,x ORDER BY MULTIPLOT,x

\PlotLarge[xlabel=,
appendLegend,
legend to name={legPlotLowCompTimeOutputSize},
legend columns=5]{dna}{%
%% MULTIPLOT(low.indextype) $lowCompTimeOutputSize('dna',)
%% CONFIG file=plot/low_timeoutputsize_dna.tex

\input{plot/low_timeoutputsize_dna.tex}
}%plot
%
\hfill
\PlotLarge[xlabel=,ylabel=]{english}{%
%% MULTIPLOT(low.indextype) $lowCompTimeOutputSize('english',)
%% CONFIG file=plot/low_timeoutputsize_english.tex

\input{plot/low_timeoutputsize_english.tex}
\legend{}
}%plot

\PlotLarge[xlabel=]{proteins}{%
%% MULTIPLOT(low.indextype) $lowCompTimeOutputSize('proteins',)
%% CONFIG file=plot/low_timeoutputsize_proteins.tex

\input{plot/low_timeoutputsize_proteins.tex}
\legend{}
}%plot
\hfill
%
\PlotLarge[ylabel=,xlabel=]{xml}{%
%% MULTIPLOT(low.indextype) $lowCompTimeOutputSize('xml',)
%% CONFIG file=plot/low_timeoutputsize_xml.tex

\input{plot/low_timeoutputsize_xml.tex}
\legend{}
}%plot
%

\PlotLarge[xlabel=]{commoncrawl}{%
%% MULTIPLOT(low.indextype) $lowCompTimeOutputSize('commoncrawl',)
%% CONFIG file=plot/low_timeoutputsize_commoncrawl.tex

\input{plot/low_timeoutputsize_commoncrawl.tex}
\legend{}
}%plot
\hfill
%
\PlotLarge[xlabel=,ylabel=]{fibonacci}{%
%% MULTIPLOT(low.indextype) $lowCompTimeOutputSize('fibonacci', AND low.indextype != '0' AND low.indextype != '1')
%% CONFIG file=plot/low_timeoutputsize_fibonacci.tex

\input{plot/low_timeoutputsize_fibonacci.tex}
\legend{}
}%plot

\PlotLarge{gutenberg}{%
%% MULTIPLOT(low.indextype) $lowCompTimeOutputSize('gutenberg',)
%% CONFIG file=plot/low_timeoutputsize_gutenberg.tex

\input{plot/low_timeoutputsize_gutenberg.tex}
\legend{}
}%plot
\hfill
%
\PlotLarge[ylabel=]{wikipedia}{%
%% MULTIPLOT(low.indextype) $lowCompTimeOutputSize('wikipedia',)
%% CONFIG file=plot/low_timeoutputsize_wikipedia.tex

\input{plot/low_timeoutputsize_wikipedia.tex}
\legend{}
}%plot
%

%% UNDEF lowCompTimeOutputSize

\ref{legPlotLowCompTimeOutputSize}

\caption{Compression time versus the compression overhead, i.e., 
the ratio of the final compressed size over the classical \LZEight{} output size minus~1. We clipped the plots at 100\% compression overhead, where the output size is twice the classical \LZEight{} output size.}
\label{figPlotLowCompTimeOutputSize}
\end{figure}

\end{document}

